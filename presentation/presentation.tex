\documentclass{beamer}

\usepackage[frenchb]{babel}
\usepackage[T1]{fontenc}
\usepackage[utf8x]{inputenc}
\usetheme{metropolis}

\author{Sylvain Ribstein}
\date{Typage, \today}
\title{Typage Recursif}
 
\institute[P7]{
  Master 2 de Recherche en Informatique 
  \\Paris 7 - Diderot}

\usepackage{listings}
\lstdefinestyle{ocaml}{
  language=caml,
  stepnumber=1,
  numbersep=8pt,
  tabsize=2,
  showspaces=false,
  showstringspaces=false,
  keywordstyle = \color{red}
}
\lstset{basicstyle=\tiny,style=ocaml}

\usepackage[framemethod=TikZ]{mdframed}

\usepackage{tikz}
\usetikzlibrary{arrows,snakes,backgrounds}
\tikzset{
  [->,
  >=latex',
  shorten >=1pt,
  auto,
  node distance=3cm,
  semithick,
  every state/.style={fill=red,draw=none,text=white}]}

\tikzstyle{transform} = [rectangle,draw=blue!50,thick, inner sep = 2pt, minimum size = 10mm]

\AtBeginSection[]{
  \begin{frame}{Sommaire}
  \small \tableofcontents[currentsection, hideothersubsections]
  \end{frame} 
}
\begin{document}

\frame{\titlepage}

% \begin{frame}
%   \frametitle{Table des matières}
%   \tableofcontents
% \end{frame}

\begin{frame}[fragile]
  \frametitle{LangAST}
  \begin{mdframed}[roundcorner=20pt,backgroundcolor=gray!50]
    \begin{lstlisting}
      type exp =
        | Var   of var 
        | Const of const
        | Pair  of exp * exp
        | App   of exp * exp
        | Abs   of var * exp
        | Let   of var * exp * exp

      and const = 
        | Fct  of string
        | Bool of bool
        | Int  of int 

      and var = string 
        
      and t = exp
    \end{lstlisting}
  \end{mdframed}
\end{frame} 

\begin{frame}[fragile]
  \frametitle{TypeAST}
  \begin{mdframed}[roundcorner=20pt,backgroundcolor=gray!50]
    \begin{lstlisting}
      type ty =
        | TInt | TBool
        | Ground of ground
        | Cross of ty * ty
        | Arrow of ty * ty
        | Rec of ground * ty

      and ground = string
      
      and t = ty
    \end{lstlisting}
  \end{mdframed}
\end{frame} 
\begin{frame}
  \frametitle{Generation de Contrainte}
  Parcours en profondeur de l'expression :
  \begin{itemize}
  \item environnement des variables introduites 
  \item type expecté de l'élément courant 
  \end{itemize}
\end{frame}

\begin{frame}[fragile]
  \frametitle{Generation de Contrainte}
  \begin{mdframed}[roundcorner=20pt,backgroundcolor=gray!50]
    \begin{lstlisting}
      let system_equation p =
        let rec aux env sys_eq expected= function
          | Var x ->
            let x_t =
              try (List.assoc x env)
              with _ -> raise (Unbound x)
            in
            (x_t, expected)::sys_eq
          | Const (Fct f)  -> (List.assoc f env, expected)::sys_eq
          | Const (Bool _) -> (TBool, expected)::sys_eq
          | Const (Int _)  -> (TInt,  expected)::sys_eq
          | Pair (m, n)    ->
            let m_t = fresh_ground () in
            let n_t = fresh_ground () in
            let sys_eq = aux env sys_eq m_t m in
            let sys_eq = aux env sys_eq n_t n in
            (Cross(m_t, n_t),expected)::sys_eq

    \end{lstlisting}
  \end{mdframed}
\end{frame} 

\begin{frame}[fragile]
  \frametitle{Generation de Contrainte}
  \begin{mdframed}[roundcorner=20pt,backgroundcolor=gray!50]
    \begin{lstlisting}
          | App(func, arg) ->
            let arg_t = fresh_ground () in
            let func_t = Arrow(arg_t,expected) in
            let sys_eq = aux env sys_eq func_t func in
            aux env sys_eq arg_t arg
          | Abs(x, m)      ->
            let x_t    = fresh_ground ()  in
            let m_t    = fresh_ground ()  in
            let env    = (x, x_t)::env        in
            let sys_eq = aux env sys_eq m_t m in
            (Arrow(x_t, m_t), expected)::sys_eq
          | Let (x, m, n)  ->
            let x_t    = fresh_ground ()  in
            let sys_eq = aux env sys_eq x_t m in
            let env    = (x, x_t)::env        in
            aux env sys_eq expected n
        in
        let final_type = fresh_ground () in
        final_type, (aux fct_type [] final_type p)
    \end{lstlisting}
  \end{mdframed}
\end{frame} 

\begin{frame}
  \frametitle{MGU avec type recursif}
  Pour chaque équation de contrainte généré :
  \begin{itemize}
  \item les règles de transformation sont appliquées
  \item les substitutions introduites sont appliquées au restes des
    équations
  \end{itemize}
  L'ensemble des substitutions sont renvoyées.
  
  type recursif 
  \begin{itemize}
  \item Si occurs-check alors introduction d'un type récursif
  \item Si un type recursif alors le type est déplié et l'unification
    reprend
  \end{itemize}
\end{frame}

\begin{frame}[fragile]
  \frametitle{MGU avec type recursif}
  \begin{mdframed}[roundcorner=20pt,backgroundcolor=gray!50]
    \begin{lstlisting}
      let rec mgu_one = function
        | (t1, t2) when t1 = t2 -> []
        | (Ground x, Ground y) ->  [(x, Ground y)]
        | (TBool, Ground x) | (Ground x, TBool) -> [(x, TBool)]
        | (TInt, Ground x) | (Ground x, TInt ) -> [(x, TInt)]
        | (Arrow (x, y), Arrow (x', y')) -> mgu [(x, x');(y, y')]
        | (Cross (x, y), Cross (x', y')) -> mgu [(x, x');(y, y')]
        | (Ground x, t) | (t, Ground x) ->
          if (Type.occurs x t) then [(x, Rec (x, t))]
          else [(x, t)]
        | Rec (x, x_t), t
        | t, Rec (x, x_t) ->
          let x_t = apply [(x, Rec (x, x_t))] x_t in
          mgu_one (x_t, t)
        | t1, t2 -> raise (NoTypable (t1, t2))

      and mgu = function
        | [] -> []
        | (x, y)::t ->
          let sub_tl = mgu t in
          let sub_hd = mgu_one ((apply sub_tl x), (apply sub_tl y)) in
          sub_hd @ sub_tl
    \end{lstlisting}
  \end{mdframed}
\end{frame} 

\end{document}